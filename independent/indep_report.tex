\documentclass[letterpaper,12pt]{article}
\usepackage[utf8]{inputenc}

\usepackage{rotating}
\usepackage[top=1in, bottom=1in, left=1in, right=1in]{geometry}
\usepackage{graphicx}
\usepackage[numbers,square,sort&compress]{natbib}
\usepackage{setspace}
\usepackage[cdot,mediumqspace,]{SIunits}
\usepackage{hyperref}
\usepackage{mathtools}
\usepackage{url}
\usepackage{authblk}
\usepackage{placeins}
\usepackage{float}

\onehalfspacing
\title{Functional Analysis: Interpolation}
\author{Anita Bahmanyar}
\affil{\small {Student Number: 998909098}}
\affil{\small {anita.bahmanyar@mail.utoronto.ca}}
\date{December 19, 2014}

\usepackage{graphicx}

\renewcommand\thesubsection{\alph{subsection}}

\begin{document}

\maketitle

\section{Introduction}
Most of the times in physics we are often given a few data points by sampling and experimenting and we need to estimate the values of our model between these points. The method commonly used to solve this problem in numerical analysis is interpolation. This is only an estimation. Interpolation is passing a curve from all the data points. There is one other case where interpolation becomes handy. For instance, we have a very complicated function to evaluate and if we want to evaluate it at all the points we want, it would take a long time to compute it and it is not efficient. In that case, we can take advantage of interpolation to estimate the value of the function at the points we want given a few points. This estimation for such a complex function might not be very accurate but the amount of computational time we save is much better than the accuracy we lose.

\section{Interpolation Types}
\subsection{Linear Interpolation}
Linear interpolation is the simplest way of getting the values of a function in between points. Linear interpolation is a method to fit a curve to the points using linear functions. If we know the coordinates of two points as $A=(x_0,y_0)$ and $B=(x_1,y_1)$, the linear interpolation between these two points would simply be a straight line. Figure 1 shows the schematic view of the interpolation.

%figure 1
\FloatBarrier
\begin{figure}[H]
\centering
\includegraphics[scale=0.5]{linear_plot.png}
\caption{Linear interpolation gives the value of point $C=(x_2,y_2)$.}
\end{figure}
\FloatBarrier


\subsubsection{Mathematical Approach}



The equation of a straight line is given by:
\begin{equation}
y = y_0 +\frac{(x-x_0)(y_1-y_0)}{x_1-x_0}
\end{equation}
Therefore, having the values of the points A and B, we can compute the value of $y_2$ given the value of $x_2$.
This can be considered as the weighted average. The weights are inversely related to the distance from the end points to the unknown point; the closer point has more influence than the farther point. 
Since this method only returns equation of lines in between each two points, the more the number of points available, the better the linear interpolation would be.
The points are simply joined by straight line segments resulting in discontinuities at each point. 

\subsubsection{Example}

Figure 2 shows the $\sin(x)$ function and the interpolated curve using 10 knots both using my linear interpolation code and scipy 1d linear interpolation. Both interpolations are almost the same showing that my code is as good as scipy. The more knots we add, the better the interpolated values demonstrate the sin function. Figure 3 shows the same curves as figure 2 with 20 knots. We can see that the interpolated values are getting closer to the $\sin(x)$ function.




%figure 2
\FloatBarrier
\begin{figure}[H]
\centering
\includegraphics[scale=0.5]{lin_sin.png}
\caption{The interpolated values from my code and scipy module are very similar that they go on top of one another. This included 10 knots.}
\end{figure}
\FloatBarrier
The interpolated values using Scipy linear 1d interpolation is very similar to the values I got from my linear interpolation code.The mean value of the difference between the Python linear interpolation and my interpolation method is $\sim 1.735 \times 10^{-17}$. Linear interpolation just gives us an overall shape of the function, it is not very accurate since it only uses straight lines between each points, so it would not be a smooth function. However, as we add more knots it would seem smoother and more similar to the actual function.

%figure 2
\FloatBarrier
\begin{figure}[H]
\centering
\includegraphics[scale=0.5]{lin_sin_morepoints.png}
\caption{The interpolated values from my code and scipy module are very similar that they go on top of one another. This is similar to previous figure except this is with more knots. This has 20 knots rather than 10 in the previous figure. We can see that the interpolated curves are getting closer to the function.}
\end{figure}
\FloatBarrier


\subsubsection{Error Analysis}
The error on linear interpolation is of order of $O(h^2)$. It has the form of:
\begin{equation}
f(x)-P_1(x) = \frac{(x-x_0)(x-x_1)}{2}f''(c_x)
\end{equation}
where $P_1(x)$ is the interpolation function in the interval $[a,b]$ and $c_x$ is a point in this interval.  This can be obtained writing $f(a)$ and $f(b)$ using Taylor series approximation and writing $P(x) = \frac{(x-x_1)}{x_0-x_1}f(x_0) + \frac{(x-x_0)}{(x_1-x_0)}f(x_1)$. Then we can substitute for $f(a)$ and $f(b)$ in $P(x)$ and get the following:
\begin{equation}
f(x)-P_1(x) = \frac{1}{2}(x-b)(x-a)f''(x)+...
\end{equation}
To give a rough estimate of the error in the $\sin(x)$ function using 20 knots and 30 points in between the knots, take the following assumptions:
\\$a=0$
\\$b=0.34210526$
\\$u_x=0.23344828$
\\Therefore, the error would be $\sim 0.003$

\subsection{Cubic Spline Interpolation}
Cubic spline interpolation is a more accurate and advanced way of fitting a curve to data points than the linear interpolation. We need to make sure the piecewise curves passing through the points have continuous second derivative at the knots. For this purpose, we need to use polynomials that are of order 3 or higher and that is where cubic spline name comes from. \subsubsection{Mathematical Approach}
The aim is to fit a curve passing through points $(x_i,y_i)$ where $i=0,1,...,n$, so we interpolate between points $(x_{i-1},y_{i-1})$ and $(x_i,y_i)$ with polynomials $y_i = q_i(x)$.


Splines take a shape so that to minimize the bending of the curve between each two knots. We require the following conditions:
\begin{equation}
q'_i(x_i) = q'_{i+1}(x_i) 
\end{equation}

\begin{equation}
q''_i(x_i) = q''_{i+1}(x_i) 
\end{equation}
for all i, $1\leq i \leq n-1$.

%equation
\begin{equation}
\frac{k_{i-1}}{x_i - x_{i-1}} +    \left (    \frac{1}{x_i - x_{i-1}} + \frac{1}{x_{i+1}-x_i} \right ) 2k_i + \frac{k_{i+1}}{x_{i+1}-x_i} = 3  \left(  \frac{y_i - y_{i-1}}{(x_i - x_{i-1})^2} + \frac{y_{i+1}-y_i}{( x_{i+1}-x_i )^2} \right)
\end{equation}
for $i=1,2,...,n-1$. This gives us $n-1$ equations including $k_0$, $k_1$, ..., $k_n$.

For the two knots at both ends, the condition is different and we have:
%equation
\begin{equation}
\frac{2}{x_1-x_0}k_0 + \frac{1}{x_1-x_0}k_1 = 3\frac{y_1-y_0}{(x_1-x_0)^2}
\end{equation}


%equation
\begin{equation}
\frac{1}{x_n-x_{n-1}}k_{n-1} + \frac{2}{x_n-x_{n-1}}k_n = 3\frac{y_n-y_{n-1}}{(x_n-x_{n-1})^2}
\end{equation}

Equations 3 and 4 give us two more equations and along with the previous $n-1$ equations we would have $n+1$ equations to solve for $k_0$,$k_1$,..,$k_n$ values. The cubic spline�s coefficients can be found by solving a tridiagonal linear system. 
%equation, matrix form
\begin{equation}
\begin{bmatrix}
a_{11} & a_{12} & &  &  & &\\ 
a_{21} & a_{22} & a_{23} & &  & &\\ 
 & a_{31} & a_{32} & a_{33} & & &\\
 &  & \ddots & \ddots & \ddots& &\\ 
 &  & & a_{n-1n-2}& a_{n-1n-1}&a_{n-1n}\\ 
 &  & & & a_{nn-1} & a_{nn}  
\end{bmatrix}
\begin{bmatrix}
k_0\\ 
k_1\\ 
k_2\\ 
\vdots\\ 
k_{n-1}\\
k_n\\
\end{bmatrix} = 
\begin{bmatrix}
b_0\\ 
b_1\\ 
b_2\\ 
\vdots\\ 
b_{n-1}\\
b_n \\
\end{bmatrix}
\end{equation}

Then having the values of $k_i$, we can calculate $a_i$ and $b_i$ values:
%equation
\begin{equation}
a_i = k_{i-1} (x_i-x_{i-1}) - (y_i - y_{i-1})
\end{equation}

%equation
\begin{equation}
b_i = -k_i (x_i - x_{i-1}) + (y_i - y_{i-1})
\end{equation}
and use these values to calculate $q_i$ values:

%equation
\begin{equation}
q_i = (1-t)y_{i_1} + ty_i + t(1-t) (a_i (1-t) +b_it)
\end{equation}
where t is:

%equation
\begin{equation}
t = \frac{x-x_{i-1}}{x_i - x_{i-1}}
\end{equation}

This will give us a function q for the points between each two knots, so we will have n different functions for the n+1 knots.

\subsubsection{Example}
Figure  4 shows a prime example of using cubic spline method to get the values using only 5 arbitrary knots.
\\As you can see in figure 5, the Python built-in interpolation function does a better job in cubic spline interpolation than my code. The zoomed in figure shows that my interpolation is still very close to the actual function even though it differs a bit from the function and the python interpolated function.


%figure 4, arbitrary cubic interpolation
\FloatBarrier
\begin{figure}[H]
\centering
\includegraphics[scale=0.5]{interp_arbitrary.png}
\caption{Interpolation of 5 arbitrary points using my code for cubic spline interpolation.}
\end{figure}
\FloatBarrier

%figure 5
\FloatBarrier
\begin{figure}[H]
\centering
\includegraphics[scale=0.09]{cubic_compared_final.png}
\caption{Upper: Knots are shown in green dots, red is the actual sin(x) function and blue and green are my interpolated function and Python built-in interpolation function, respectively. Lower: Zoomed in version of the figure above. Python built-in function does a better job in cubic spline interpolation.}
\end{figure}
\FloatBarrier

\subsubsection{Error Analysis}
The error in cubic spline interpolation is of order $O(h^4)$. The error would be the maximum difference between the value of the function we are approximating and our interpolated value. The error in cubic spline is given by:
\begin{equation}
\left | s-f \right | \sim \frac{5}{384}. \left | f^{(4)} \right |. h^4
\end{equation}
where $s-f$ is the difference between interpolated and the function value, $f^{(4)}$ is the value of fourth derivative of the function and h is the spacing between knots.
For example, if we do the interpolation on the function $y(x) = \sin(x)$ over the interval [0,3] with $h$= 0.1, what would the error be?
\\We know $f^{(4)}(x)$ = $\sin(x)$; therefore, $f^{(4)}(x) \leq$ 1. So the value of the error would be:
\\ $\frac{5}{384}.1.(0.1)^4$ $\approx$ $\frac{1}{768000} \approx 1.3 \times 10^{-6}$. This error is very small so the cubic spline interpolation is a good approximation of the function. 

Now to see how good it works, I have applied the code to do the interpolation for the function $\sin(x)$ in the interval [0,6.5] with 20 points, so $h$ = 0.325 here. So the error would be $\sim 1.46 \times 10^{-4}$. This is still quite a small error.

\subsection{Bilinear Interpolation}
Bilinear interpolation is the same as linear interpolation which is extended into two dimensions. Unlike 1D interpolation where you entered $x$ values and got $y=f(x)$ values out, this would require you to enter $x$ and $y$ values to get the value of the function $z = f(x,y)$ out. Figure 5 shows a schematic view of bilinear interpolation.

%figure 5
\FloatBarrier
\begin{figure}[H]
\centering
\includegraphics[scale=0.5]{bilinear_plot.png}
\caption{The red points show the knots and point P which is where we want to calculate the value is shown by green dot. }
\end{figure}
\FloatBarrier


\subsubsection{Mathematical Approach}
The four points indicated by red dots in figure 5 have the values of:
\\$Q_{11} = (x_1,y_1)$, $Q_{12} = (x_1,y_2)$, $Q_{21} = (x_2,y_1)$ and $Q_{22} = (x_2,y_2)$. Also, the blue dots in figure 5 are $R_1=(x,y_1)$ and $R_2=(x,y_2)$. The green dot is the point where we want to find the value $P=(x,y)$.
\\Doing linear interpolation in x-direction first would give us:
\begin{equation}
f(R_1) \approx \frac{x_2-x}{x_2-x_1}f(Q_{11}) + \frac{x-x_1}{x_2-x_1}f(Q_{21})
\end{equation}
and
\begin{equation}
f(R_2) \approx  \frac{x_2-x}{x_2-x_1}f(Q_{12}) + \frac{x-x_1}{x_2-x_1}f(Q_{22})
\end{equation}
Then we continue the interpolation in y direction which gives us:
\begin{equation}
f(P) \approx \frac{y_2-y}{y_2-y_1}f(R_1) + \frac{y-y_!}{y_2-y_1}f(R_2)
\end{equation}
Finally, the estimate of the interpolated function would be:
\begin{eqnarray}
f(x,y) \approx \frac{1}{(x_2-x_1)(y_2-y_1)} \left ( f(Q_{11})(x_2-x)(y_2-y) + 
\\f(Q_{21})(x-x_1)(y_2-y) + 
\\f(Q_{12})(x_2-x)(y-y_1) + 
\\f(Q_{22})(x-x_1)(y-y_1) )\right)
\end{eqnarray}
This could have been done by doing the linear interpolation first in y direction and then in x direction without any change in the result. This kind of interpolation is linear in x and in y direction separately but not as a whole despite its name. Figure 6 shows how bilinear interpolation method works. We multiply the area of each square by the value of the knot on the opposite side and divide the result by the area of the whole square ABCD to normalize the vaues.

%figure of colours
\FloatBarrier
\begin{figure}[H]
\centering
\includegraphics[scale=0.5]{bilinear_plot2.png}
\caption{The value of the function at point $P=(x,y)$ is the value of each coloured point multiplied by the area of the same colour divided by the area of ABCD rectangle to normalize it.}
\end{figure}
\FloatBarrier

\subsubsection{Example}
Figure below shows an example of running my code for bilinear interpolation. The four points given are located at the sides of a unit square and the $Q_{11}$, $Q_{12}$, $Q_{21}$ and $Q_{22}$ values are 0,1,1,0.5, respectively. i have compared the values I got for interpolation using my own code with the values from scipy 2d interpolation and the two plots seem very similar. The mean value of the absolute value of the difference between the values of the two is $\sim 1.998 \times 10^{-17}$

%contour plot
\FloatBarrier
\begin{figure}[H]
\centering
\includegraphics[scale=0.45]{table1.png}
\caption{Comparison between my bilinear interpolation code and scipy 2d interpolation using four points on a unit square. Left: Using my code. Right: Using scipy. The two figures are quite similar showing my code works fine.}
\end{figure}
\FloatBarrier


\section{Analysis}
In this section, I will test my code for different functions and compare it with scipy interpolation modules. Figure below shows the step function curve, its linear and cubic interpolation values using both my code and scipy module. The two plots show the same curves using 15 knots, 20 and 100 points in between, respectively. We can see that as we keep the number of knots the same and increase the number of points in between the knots, we get a better approximation of the step function. No matter how many points we add in between knots, there is always an effect at the edges called Gibb's phenomena which is due to the discontinuity of the step function. In the case of step function linear interpolation does a better job both in the case of my code and scipy code.

\FloatBarrier
\begin{figure}[H]
\centering
\includegraphics[scale=0.5]{step2.png}
\caption{15 knots and 20 points in between.}
\end{figure}
\FloatBarrier

\FloatBarrier
\begin{figure}[H]
\centering
\includegraphics[scale=0.5]{step5.png}
\caption{15 knots and 100 points in between.}
\end{figure}
\FloatBarrier
Also, we can see that my cubic spline interpolation code fails to fit a curve to the step function knots.

%exponential plot
\FloatBarrier
\begin{figure}[H]
\centering
\includegraphics[scale=0.45]{exponential.png}
\caption{$e^{(-4x^2)}$ along with linear and cubic spline interpolation using both my code and scipy. The linear interpolation using my code and scipy do match up exactly while the cubic spline interpolations differ.}
\end{figure}
\FloatBarrier


\section{Conclusion}
Cubic spline interpolation approximates the curve between data points much better than linear interpolation. The error in linear interpolation is of order two while the error in cubic spline interpolation is of order four. 
Generalization of the error in polynomial interpolation:
\begin{equation}
f(x) - P_n(x) = \frac{(x-x_0)(x-x_1)...(x-x_n)}{(n+1)!}f^{(n+1)}(u_x)
\end{equation}
where $u_x$ is a point in the interval of interpolation.


\section{References}
1-http://bmia.bmt.tue.nl/people/BRomeny/Courses/8C080/Interpolation.pdf
\\2-http://en.wikipedia.org/wiki/Linear\_interpolation
\\3-http://en.wikipedia.org/wiki/Bilinear\_interpolation
\\4-http://en.wikipedia.org/wiki/Spline\_interpolation
\\5-http://homepage.math.uiowa.edu/~atkinson/ftp/ENA\_Materials/Overheads/sec\_4-2.pdf
\\6-http://www-solar.mcs.st-andrews.ac.uk/~clare/Lectures/num-analysis/Numan\_chap3.pdf
\\7-http://en.wikipedia.org/wiki/Interpolation

\end{document}




